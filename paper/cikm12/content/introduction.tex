

\section{Introduction}

% increasing amounts of unstructured text data
\eat{The field of database management has traditionally focused on
structured data, providing little or no help for the significantly
larger amounts of the world's data that is unstructured.}
For many
applications, unstructured text and structured data are both
important natural resources to fuel data analysis. For example, a
sports journalist covering NFL (National Football
League\footnote{http://www.nfl.com} - 32 American football teams with
more than 1700 players) games would need a system that can analyze both the
structured statistics (e.g., scores, biographic data) of teams and players and the
unstructured tweets, blogs, and news about the games.

In such applications, analytics are performed over text data
from many sources. Text analysis uses 
statistical machine learning (SML) methods to extract structured
information, such as part-of-speech tags, entities, relations, 
sentiments, and topics from
text. The result of the text analysis can be joined with other
structured data sources for more advanced analysis. For example, a sports
journalist may want to correlate fan sentiment from tweets 
with statistics describing the player and team performance
of the Miami Dolphins\footnote{http://www.miamidolphins.com/}.

\eat{To our knowledge, there is no integrated system with a query
interface that enables domain experts (e.g., journalists) to issues
such ad hoc exploratory queries.}
To answer such queries, a software
developer is needed to understand and connect multiple tools,
including Lucene for text search, Weka or MATLAB for sentiment
analysis, and a database to joining the structured data with the
sentiment results. Using such a complex offline batch process to 
answer a single query
makes it difficult to ask ad hoc queries over ever-evolving text data. 
These queries are essential for applications, such as computational journalism,
e-discovery, and political campaign management, where queries are
exploratory in nature, and follow-up queries need to be asked based
on the result of previous queries.

% need of query-driven analysis
% higher-level of abstraction of ML models using simple ER models and SQL queries
% real-time analysis enable exploratory queries and query refinement based on result
% many applications: computational journalism, campaign management, e-discovery, etc.



{\system} implements four important text analysis functions, namely  
part-of-speech tagging, entity extraction, classification 
(e.g., sentiment analysis), and entity resolution.
Text analysis functions are developed on PostgreSQL, 
a single-threaded database, and Greenplum, a massive parallel 
processing (MPP) framework. 
Two SML models and their inference algorithms are adapted: linear-chain 
CRF and Naive Bayes\cite{Wang:2008:BML:1453856.1453896}. 
\eat{Two types of indexes are used: inverted index and trigram index.} 
In-database and parallel SML algorithms are implemented in Greenplum MPP framework.
{\system} text analytic library is integrated into the MADLib
open-source project\footnote{MADLib is an open source project for scalable in-database 
analytics \url{http://madlib.net}}.
The declarative SQL query interface with MADden text analysis functions 
provides a higher-level abstraction. Such an abstraction shields users from 
detailed text analytic algorithms and enables users to focus more on the 
application specific data explorations. 



In the demonstration of \system, we will show the 
following points using journalism on the NFL as our driving example:
\begin{itemize}[noitemsep]
\item Processing declarative ad hoc queries involving various statistical text analytic functions;
\item Joining and querying over multiple data sources with both aggregation structured and text information;
\item Query-time rendering of visualizations over query results, using 
word clouds, histograms, and ranked lists of documents.
\end{itemize}

