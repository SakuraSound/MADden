
\section{{\system } vs Industry Techniques}


To answer this question, one needs to find all the news articles where the
NFL team is mention and the player is mentioned.
We need to perform an entity resolution step -- our dataset may contain many
articles about teams from Arizona and Teams with a mascot as a Cardinal.
Next, we compute the
aggregate sentiment for the document. We can then rank the documents with
respect to the aggregate score.
Notice the question is not concerning news about the player individually
but the document as a whole. No time smoothing is required for
for accurate answers.

To answer a query as described above, a developer may use the combination of
a text search engine such as Lucene with an analytic toolkit such as
MATLAB or R. A developer would first create indexes for the data in Lucene for
the possible documents that contain the team, player and their associated
aliases. After the documents are retrieved a developer can use an off the
shelf algorithm to calculate the sentiment, then they can sort and rank each
document.
Any change in the dataset may necessitate a change in the index and
the algorithm.
Tools such as Hadoop MapReduce would have a similar batch processing
work flow.

Using {\system} we are able to perform a declarative query over the data set
to return the set of correct documents. Using native SQL statements, we get
a query optimizer that allows us to perform each step of the query with
out iteratively waiting to load all documents. Another benefit of {\system} is
we bring the query to the data, there is no needed to make large copies
spend time
transferring portions of the data around. For large data sets, the data becomes
unwieldy for in memory implementations. Using {\system}, the system and any
algorithms scale with the data.
A user does not have to rely on samples, as commonly done using analytic tools.
We provide a framework for data analyst to ingest and analyze the information
on the fly. The system that hold the data is the same system that is queries.
To support adhoc textual analysis over large data sets a query-driven
approach is necessary for real-time exploration. During
the computation we can use the information given in the query to reduce
total computation.

\ceg{Emphasize the benefits of the query interface and out model. Make the advantages more clear}
