
\section{Example Query}

In this example query we demonstrate the flexibility and multi-modal 
support of \system. Our running query is the question
``{\tt
Give me the \underline{best} new 
involving the National \underline{Arizona} \underline{Cardinals} 
National Football League team and involving the 
player \underline{Larry Fitzgerald}.}''
Notice this query has three parameters, each demarcated by underlines.
The qualifier `best' here is interpreted as news articles that have the 
highest ratio of good news to bad news.
`Arizona' Cardinals' is a professional American Football team in the National
Football League (NFL) and `Larry Fitzgerald' is a player for that team.

\begin{table}
\begin{center}
\begin{tabular}{|l|}
\hline
\multicolumn{1}{|c|}{Schema}\\
\hline
Player (id, name, ...)\\
\hline
PlayerAlias (pid, nickname)\\
\hline
Team (id, name, city)\\
\hline
TeamAlias (teamid, nickname)\\
\hline
BlogEntry (id, entry, title, posted\_at)\\
\hline
\end{tabular}
\end{center}
\caption{This is the schema for Example query}
\label{tab:schema}
\end{table}

As data we are given tables\footnote{These
tables may be extracted to an RBDMS before hand or defined over an api using
a foreign data wrapper (fdw). }
as described in Table \ref{tab:schema}.
The {\tt Player} table is a list of individuals who play in the NFL with
associated ids. The {\tt PlayerAlias} table list alternative names for players.
The {\tt Team} table holds all the team names in the NFL and the {\tt TeamAlias}
table has all their nick names and abbreviations. The {\tt BlogEntry} table
is a list of arbitrarily authored documents crawled from the web and downloaded
from news companies.



