\documentclass{article}
\usepackage[english]{babel}
\usepackage[utf8]{inputenc}
\usepackage[T1]{fontenc}
\usepackage{amsmath,amssymb,amsfonts}

\usepackage{xspace}
\usepackage{graphicx}

\newcommand{\system}{MADden\xspace}

\title{MADden : In-Database Text Analytics}
\author{Morgan Bauer 9890-4838 \\
  Christan Grant 8143-3970 \\
  Joir-dan Gumbs 6148-9357}
\date{December 13, 2011}
\begin{document}
\maketitle
\begin{enumerate}
\item Names of the group members and UFIDs

  % not sure if it's best to put this into the public with our uf-id on it ~mhb

  Morgan Bauer

  Christan Grant

  Joir-dan Gumbs

\item Title of the project and abstract

  MADden : In-Database Text Analytics

ABSTRACT



\item Final report should include the following topics:

  \section{Introduction (1 page)}

  MADlib, MAD, Magnetic Agile Deep.
  Madlibs, fill in the blank games, that usually have amusing and unforeseen outcomes.
  MADden, our initial problem domain deals with football,
  and John Madden is a famous former player, coach, and commentator,
  also known for the long-running series of video games that bear his name.

  \begin{enumerate}
  \item The problem statement and a summary of the main contributions and results of the project.

    Problem is he large amounts of structured and unstructured data that is available, and how to intermix the two.
    We use statistical text analysis techniques to extract structured information from unstructured data,
    and pass this on to later stages for further processing.

  \item What is the responsibility of each member of the group?



    Morgan did consulting work among the various group members, and between groups.
    Official NFL team blogs. NFL stats from cbssports.com.
  \end{enumerate}

  \section{Background (1 page)}
  \begin{enumerate}
  \item Background information (basic concepts, algorithms, etc.)


    algorithms -- crf (viterbi, forward-backward, linear vs skip-chain) evaluation over pre-trained model done in database.



  \item Related work

    related work

    MADlib, directly

    Twitter sentiment, for twitter sentiment.

    IIT for CRF training.

    Wang et al. for idea of doing CRF in database

  \item How is this project different from related work

    integration of analysis queries into the database. ability to build ad hoc queries, as if they fit.
  \end{enumerate}

  \section{System Description (4 pages)}
  \begin{enumerate}
  \item What is the final system architecture?

    {\system} is a four layered system, as can be seen in Figure \ref{fig:arch}.
    The user interface is where both naive and advanced users can construct queries over text, structured data, and models.
    From the user interface,
    queries are then passed to the DBMS,
    where MADLib and {\system} libraries sit on top of the query processor to add statistical and text processing functionality.
    These queries are processed using PostgreSQL/Greenplum's Parallel DB architecture to further optimize on replicated storage.


    PostgreSQL, with MADlib overtop.

    There is a new feature for MADlib that implements CRF evaluation, but not training, in the database.
    CRF specifically -

    \begin{enumerate}
    \item information extraction

      Verbatim from previous paper.

      Once we have the text documents, we would like to then be able to identify
      objects of interest. Information Extraction methods become increasingly
      important as the size of our corpus gets large. Many methods exist for in-database data
      extraction, but we are currently focused on building upon existing work in
      utilizing Conditional Random Fields (CRFs) for query-time extraction of
      entities.

    \item entity resolution

      What it is -
      This links various forms and representations of the same background object,
      so that a query for one of them returns the results for all of them.

      Why is it necessary -
      This is necessary for two reasons,
      the informality of speaking in documents like blog posts and twitter tweets,
      which leads to misspellings and slang
      and the more  general case of people or companies having many names to refer to them.

      Example of informality, slang \& misspelling. tmrw, tomorrow. NY Jest, NY Jets.

      Example of multiple names for a person. Nickname for Larry Fitzgerald, Sticky Fingers. IBM, Big Blue. New York City, Big Apple.

      This is implemented through q-grams for misspellings and alternate spellings, and an alias table for multiple names of an entity.
      q-grams is the automatic approach, with the alias table being hand constructed.

    \item sentiment analysis

      Sentiment analysis extracts the polarity of subjective opinions from documents.

    \item part of speech tagging

      Part of Speech Tagging (POS tagging) labels each token of a sequence with a tag based on the grammar of a natural language.
      These tags are parts of speech, based on context and word definitions.
      The generators for these tags can come from both supervised and unsupervised sources.
      Statistical methods are used to generate taggers specific to a corpus, and thus a particular subject.
      These parts of speech are useful in other stages, such as entity resolution and information extraction (ER \& IR).

      Using CRF -
      Initially trained on email headers.
      This data was not generic enough to be useful.
      The individual documents were to small and not of a form similar enough to ensure correct tags.

      The next CRF was trained using (email contents)? .
      While it had a higher accuracy, the training corpus was probably too small (only 100 or 200 documents right?).
      The tagging accuracy was higher than the previous CRFs

      We need to train a CRF specific to our domain,
      and would like to train a CRF specific to each type of text,
      such as a twitter CRF, a blog CRF, etc.

      Using NLTK -
      Our current system does not utilize the CRF style model.
      Instead, as a UDF, we make a call to the Python Natural Language Toolkit (NLTK).
      (what kind of tagger is this? Brill? Naive Bayes? I don't know where the sourcecode is. ~mhb)
      This returns the tagged data, which can be used in PostgreSQL for later stages of analysis.


    \end{enumerate}






    call to twitter sentiment . This is done as a udf calling python script which uses a JSON API.

    Call to nltk for entity resolution.

    \begin{figure}
      \begin{center}
        \includegraphics[width=104mm]{architecture-1.png}
        \caption{System Architecture}
        \label{fig:architecture}
      \end{center}
    \end{figure}

    \begin{figure}
      \begin{center}
        \includegraphics[scale=0.4]{arch.png}
        \caption{{\system} architecture}
        \label{fig:arch}
      \end{center}
    \end{figure}

    \begin{figure}
      \begin{center}
        \includegraphics[scale=0.3]{web-ui.png}
        \caption{{\system} Web UI}
        \label{fig:web-ui}
      \end{center}
    \end{figure}

  \item What are the system components and statistical methods developed?

  \item What are the final data products?

    We don't have particular products, so much as abilities to deliver varying products.

    Ad hoc queries

    Query results
  \end{enumerate}


  \section{Experiments (2 pages)}
  \begin{enumerate}
  \item What datasets have been used? What’s the measure of success?

    Datasets
    % my description of document size is vague ~mhb

    Twitter -- unstructured, microblogs, very small document size (140 characters).

    Blogs -- english language, small-medium document size, 10-100 sentences.

    play-by-plays -- semi-structured, repeated patterns with specific meaning.

    Measure of success?

    Do we get sane output? Yes.

  \item What are the main experimental results? Effectiveness/performance?
  \end{enumerate}

  \section{Conclusion (1 page)}
  \begin{enumerate}\item What you have learnt through this project?
    What are the difficulties you have encountered in terms of systems and algorithms in building the system to deliver the data products?
    What kinds of Data Science tools/systems are needed?
  \end{enumerate}
\end{enumerate}

References

MADlib http://madlib.net/

https://sites.google.com/site/twittersentimenthelp/

IIT CRF http://crf.sourceforge.net/
\end{document}

%%% Local Variables:
%%% mode: latex
%%% TeX-master: t
%%% End:
